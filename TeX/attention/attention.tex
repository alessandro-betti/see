\input colordvi
\magnification=\magstep1
\baselineskip14pt
\tracingpages=1 % show paging decisions, to help with figure placement
\baselineskip=12pt minus .1pt % allow tiny bit of flexibility

\def\today{\ifcase\month\or
  January\or February\or March\or April\or May\or June\or
  July\or August\or September\or October\or November\or December\fi
  \space\number\day, \number\year}

\newcount\twodigits
\def\hours{\twodigits=\time \divide\twodigits by 60 \printtwodigits
  \multiply\twodigits by-60 \advance\twodigits by\time
  :\printtwodigits}

\def\gobbleone1{}
\def\printtwodigits{\advance\twodigits100
  \expandafter\gobbleone\number\twodigits
  \advance\twodigits-100 }
  
\def\frac#1#2{{#1\over #2}}
\def\feat{C\mskip -10mu\lower-2pt\hbox{\fiverm 1}\,}
\def\dts{\mathinner{\ldotp\ldotp}}
\def\ast{\mathop{\hbox{\lower 1.5pt\hbox{$\buildrel x\over *$}}}}
\def\Ast#1{\mathop{\hbox{\lower 1.5pt\hbox{$\buildrel #1\over *$}}}}
\def\tr{\mathop{\rm tr}}
\def\vec{\mathop{\rm vec}}

\font\ninerm=cmr9
\font\eightrm=cmr8

%star heading
\font\manfnt=manfnt % special symbols from the TeX project
\font\titlefont=cmssbx10 scaled\magstep2
\def\volheadline#1{\line{\cleaders\hbox{\raise3pt\hbox{\manfnt\char'36}}\hfill
       \titlefont\ #1\ \cleaders\hbox{\raise3pt\hbox{\manfnt\char'36}}\hfill}}
%end

% macros for verbatim scanning
% the max number of \tt chars/line is 66 (10pt), 73 (9pt), 81 (8pt), 93 (7pt)
% minus 4 or 5 characters for indentation at the left
\chardef\other=12
\def\ttverbatim{\catcode`\\=\other
  \catcode`\{=\other
  \catcode`\}=\other
  \catcode`\$=\other
  \catcode`\&=\other
  \catcode`\#=\other
  \catcode`\%=\other
  \catcode`\~=\other
  \catcode`\_=\other
  \catcode`\^=\other
  \obeyspaces \obeylines \tt}
\def\begintt{$$\ttverbatim \catcode`\|=0 \ttfinish}
{\catcode`\|=0 \catcode`|\=\other % | is temporary escape character
  |obeylines   % end of line is active
  |gdef|intt#1^^M{|noalign{#1}}%
  |gdef|ttfinish#1^^M#2\endtt{#1|let^^M=|cr %
    |halign{|hskip|parindent##|hfil|cr#2}$$}}

\line{{\bf Incorporating the focus of attention}\hfil{\eightrm
rough notes, \today\/ @ \hours}}
\bigskip
\bigskip\begingroup\narrower\narrower
\noindent Assuming that we are in posses of an attention
trajectory we will reformulate the theory to seriously take into
account this.
\par\endgroup

\bigskip
\bigskip\noindent
The features are as always extracted by convolutional processing of the
input signal:
$$\feat_i(x,t)=\sigma_i(A_1,\dots,A_n); \qquad A_i(x,t):=\sum_{j=1}^m\int_X
\varphi_{ij}(x-\xi,t)C_j(\xi,t)\, d\xi.$$
Once discretized the mutual information will go as usual into a potential,
we just need to take care of the motion invariance term.

We require the invariance under motion on the trajectory of attention
$t\mapsto a(t)\in X$; that is to say we impose that the functions
$\alpha_i(t):=A_i(a(t),t)$ are such that $\dot \alpha_i=0$.

The corresponding term in the action is
$$\int_0^T e^{\theta t} v_m(t)\, dt; \qquad v_m(t)=\frac{1}{2}
\sum_{i=1}^n(\partial_t A_i(a(t),t)+\nabla_x A_i(a(t),t))^2.$$
Assuming that the convolution is invertible 
we then have
$$v_m(t)=\frac{1}{2}
\sum_{i=1}^n\biggl(
\sum_{j=1}^m\int_X\Bigl(\partial_t \varphi_{ij}(\xi,t) C_j(a(t)-\xi,t)
+\varphi_{ij}(\xi,t){d\over dt}C_j(a(t)-\xi,t)\Bigr) \biggr)^2,$$
or written on a discretized receptive field of size $\ell\times\ell$:
$$v_m(t)=\frac{1}{2}
\sum_{i=1}^n\biggl(
\sum_{j=1}^m\sum_{\xi_1,\xi_2=1}^\ell
\Bigl(\dot \varphi_{ij\xi_1\xi_2}(t) \gamma_{j\xi_1\xi_2}(t)
+\varphi_{ij\xi_1\xi_2}(t)\dot\gamma_{j\xi_1\xi_2}(t)\Bigr) \biggr)^2,$$
where $\gamma_{j\xi_1\xi_2}(t):=C_{j(a_1(t)-\xi_1)(a_2(t)-\xi_2)}(t)$.
Expanding the square we thus obtain
$$\eqalign{
v_m={1\over 2}\sum_{i=1}^n\sum_{j,k,\xi_1,\xi_2,z_1,z_2}
\bigl(&\dot \varphi_{ij\xi_1\xi_2} \gamma_{j\xi_1\xi_2}\dot
\varphi_{ik z_1z_2} \gamma_{kz_1z_2}
+\varphi_{ij\xi_1\xi_2} \dot\gamma_{j\xi_1\xi_2}
\varphi_{ik z_1z_2} \dot\gamma_{kz_1z_2}\cr
&+2\dot \varphi_{ij\xi_1\xi_2} \gamma_{j\xi_1\xi_2}
\varphi_{ik z_1z_2} \dot\gamma_{kz_1z_2}\bigr).\cr}$$
If we now define
$$
Q:=\pmatrix{\varphi_{1111}&\varphi_{1112}&\cdots&\varphi_{1m\ell\ell}\cr
           \varphi_{2111}&\varphi_{2112}&\cdots&\varphi_{2m\ell\ell}\cr
           \vdots&\vdots&\cdots&\vdots\cr
           \varphi_{n111}&\varphi_{n112}&\cdots&\varphi_{nm\ell\ell}},
           \quad\hbox{and}\quad \gamma:=(\gamma_{111},\gamma_{112},\dots,\gamma_{m\ell\ell}).
$$
With this notation we have
$$\eqalign{
v_m&={1\over2}\sum_{ijk}\Bigl(\dot Q_{ij}(\gamma\otimes\gamma)_{jk}
\dot Q_{ik}+Q_{ij}(\dot\gamma\otimes\dot\gamma)_{jk}
Q_{ik}+2\dot Q_{ij}(\gamma\otimes\dot\gamma)_{jk}
Q_{ik}\Bigr)\cr
&{1\over 2}\Bigl(\tr(\dot Q\gamma\otimes\gamma \dot Q')
+\tr(Q\dot\gamma\otimes\dot\gamma Q')
+2\tr(\dot Q\gamma\otimes\dot\gamma Q')\Bigr).\cr}$$
Alternatively if we let $q:=\vec{Q}$ (row vectorization of Q), then
$$v_m={1\over 2}\biggl(\dot q\cdot \Bigl(\bigoplus_{i=1}^n
\gamma\otimes\gamma\Bigr) \dot q
+q\cdot \Bigl(\bigoplus_{i=1}^n\dot\gamma\otimes\dot\gamma\Bigr) q
+2\dot q\cdot\Bigl(\bigoplus_{i=1}^n\gamma\otimes\dot\gamma\Bigr) q\biggr)$$
This means that by comparison with calculations done in
the arxiv report  we have that
$$M^\natural \gets \bigoplus_{i=1}^n\gamma\otimes\gamma,\quad
O^\natural\gets \bigoplus_{i=1}^n\dot\gamma\otimes\dot\gamma\quad
\hbox{and}\quad N^\natural\gets\bigoplus_{i=1}^n\dot \gamma\otimes \gamma.$$



\bye