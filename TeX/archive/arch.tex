\font\ninett=cmtt9
\font\eighttt=cmtt8 
\def\uncatcodespecials{\def\do##1{\catcode`##1=12 }\dospecials}

 \newcount\lineno % the number of file lines listed
        \def\setupverbatim{\eighttt \lineno=0
        \def\par{\leavevmode\endgraf} \catcode`\`=\active
          \obeylines \uncatcodespecials \obeyspaces
          \everypar{\advance\lineno by1 \llap{\sevenrm\the\lineno\ \
        }}}
        {\obeyspaces\global\let =\ } % let active space = control space

\def\listing#1{\par\begingroup\setupverbatim\input#1 \endgroup}

\def\scriptdir{../../videomotion/run_scripts}

\def\meno{\medskip\noindent}

\centerline{\bf Experiment archive}
\smallskip
\centerline{SaiLab}
\bigskip

\noindent{\bf Ex 1.\enspace}
We tried to choose the parameters in such
a way that the roots of the characteristic polynomial has stable and
real roots. We found that if we choose
$$\alpha=1, \quad\beta=6.25\cdot10^{-10},\quad\gamma=1.25\cdot10^{-5},
\quad k=10^{-20},\quad \theta=10^{-4};$$
we have the following roots
$$\lambda_1\approx\lambda_2\approx -10^{-4},\qquad \lambda_3\approx -6\cdot
10^{-6},\qquad \lambda_4\approx -2\cdot 10^{-7}.$$
\medskip
\listing{\scriptdir/exp1.sh}
\medskip
Observations....

However we also notice that it develops a solution which makes the norm of
$q$ very large. We suspect that this is due to the growing dynamic of
$\rho(t)$.

\meno
{\bf Ex 2.\enspace} What if we penalize the growth of $\Vert q\Vert^2$
taking a bigger $k=0.001$ (all the other parameters are the same as before).
This gives instability, however since we have
the night-reset we hope to handle this situation.
$$\lambda_1=\bar \lambda_2\approx -0.1(1+i),\qquad \lambda_3=\bar\lambda_4
\approx 0.1(1-i).$$
\medskip
\listing{\scriptdir/exp2.sh}
\meno
This experiment develops similar features with respect to the previous
set.However there seems to be a little more oscillating but after some time
it reaches a very stable configuration (the norm of $q$ and a random component
of $q$ becomes constant).

\bye