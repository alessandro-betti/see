\def\meno{\medskip\noindent}

\centerline{\bf Algebraic analysis of our characteristic equation}
\meno
The choice of the regularization parameters of our theory determines the
nature of the roots of the roots of the characteristic polynomial of our forth
order equation in the absence of the input.

Consider a quartic equation
$$P(x):=  x^4+b x^3 +c x^2+d x+e=0, \eqno(1)$$
performing the substitution $x=z-b/4$ we obtain the reduced quartic equation
$$Q(z):=P(z-b/4)=z^4+q z^2+r z+s=0,\eqno(2)$$
where
$$q=c-{3 b^2\over 8}, \quad r={b^3\over 8}-{bc\over 2}+d,\quad s=
{b^2c\over 16}-{3\over 256} b^4-{bd\over 4}+e. \eqno(3)$$

\proclaim
Theorem A {\rm ({\it Lagrange\/})}. A quartic equation~(2) with
q,r,s, real, $r\ne 0$, and with discriminant $\Delta$, has
\medskip
\item{1. } 4 distinct real roots if $q<0$, $4s-q^2<0$ and $\Delta>0$;
\item{2. } no real root if $q\ge0$ and $4s-q^2\ge 0$ and $\Delta>0$;
\item{3. } 2 distinct real and 2 distinct imaginary roots if $\Delta<0$;
\item{4. } at least 2 equal real roots if $\Delta=0$.

Since we are interested in reality we should analyze also point 4. in
some details. The following result completes the characterization of
real roots:

\proclaim
Theorem B.  A quartic equation~(2) with
q,r,s, real, $r\ne 0$, and with discriminant $\Delta\equiv 0$, has
\medskip
\item{1. } two real roots 


\bye